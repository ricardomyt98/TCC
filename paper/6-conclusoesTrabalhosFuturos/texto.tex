%----------------------------------------------------------------------------------------
%	Conclusões
%----------------------------------------------------------------------------------------
\chapter{Conclusões}
É interessante e útil ver o comportamento e a distribuição dos valores de centralidade dos caminhos mínimos nos grafos selecionados, pois, pelos gráficos obtidos, constata-se que os grafos possuem um comportamento muito parecido entre eles, segundo a relação de caminhos mínimos.

Vale ressaltar que o algoritmo proposto neste trabalho para o problema de caminhos mínimos é custoso, podendo-se tornar inviável para algumas aplicações da vida real, porém este trabalho visa despertar a curiosidade referente a importância e utilidade deste tipo de análise.

Da análise dos valores de centralidade dos caminhos mínimos: de forma intuitiva, já se percebe um comportamento e relação entre o tamanho do percurso de um caminho mínimo e seu valor de centralidade. Em outras palavras, caminhos mínimos com um percurso maior tendem a ter valores de centralidade menores. Ou seja, dado um caminho mínimo, ele só poderá estar contido como subcaminho em algum outro, se e somente se, este conter pelo menos a mesma quantidade de vértices; isto é, quanto menor o percurso, mais possibilidades deste caminho estar contido como subcaminho mínimo em outros e, consequentemente, maior o seu valor de centralidade. Em resumo, pode-se dizer que, quanto maior o percurso de um caminho mínimo, menor seu valor de centralidade e, de forma análoga, quanto menor o percurso de um caminho mínimo, mais provável de seu valor de centralidade ser maior.

Do comportamento do gráfico de distribuição dos valores de centralidade dos caminhos mínimos: o Gráfico~\ref{sec5:short_path_centrality_ex2} e o Gráfico~\ref{sec5:short_path_centrality_ex3} apresentam a disposição dos valores de centralidade ordenados de forma crescente, permitindo-nos constatar que a curva é sempre crescente. Isto é importante destacar, pois indica que os caminhos mínimos tendem a ter comportamentos extremos (ser muito centrais -- "importantes" -- ou pouco centrais). Outro fator interessante disto, assim como supracitado, é o de que os caminhos mínimos dispostos mais à esquerda (valores menores), referem-se aos caminhos mínimos com percurso maior, enquanto que, de forma análoga, os caminhos mínimos dispostos mais à direita dos gráficos representam os percursos maiores.

Vale ressaltar um aspecto em adicional. Sabemos que existem diversas situações em que, dado um grafo, existe a possibilidade de ocorrer mais de uma opção de caminho mínimo entre um par de vértices. Isto é um aspecto importante, pois, de forma lógica, os valores de centralidade possivelmente mudem. Em grafos pequenos este aspecto pode ser facilmente percebido e por isso é necessário assumir como verdade o retorno de um algoritmo de busca por caminhos mínimos, pois cada algoritmo pode tratar a busca. Além disso, existem algumas linhas de pesquisas \cite{alane2021} que indicam que o problema da aleatorização de rotulagem tem impacto desprezível em grafos muito esparsos. Entende-se disto que a aleatorização de rotulagem não é, necessariamente, um empecilho para o estudo da centralidade de caminhos mínimos em um grafo.