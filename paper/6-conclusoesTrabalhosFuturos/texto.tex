%----------------------------------------------------------------------------------------
%	Conclusões
%----------------------------------------------------------------------------------------
\chapter{Conclusões}
Neste trabalho, foram usados três tipos de grafos de entrada, um deles é muito simples e não reflete situações da vida real, que são os tipos de aplicações que nos interessam. O propósito do uso de um grafo mais simples neste trabalho é permitir o entendimento mais claro do algoritmo para a computação da centralidade de caminhos mínimos proposto. Já os outros dois grafos, são modelos que refletem condições da vida real, dito isto, podem ser inviáveis ou muito difíceis de analisar visualmente ou por \emph{teste de mesa}. É interessante e útil ver o comportamento e a distribuição dos valores de centralidade dos caminhos mínimos nos grafos selecionados, pois, pelos gráficos obtidos, constata-se que os grafos possuem um comportamento muito parecido entre eles, segundo a relação de caminhos mínimos.

Vale ressaltar que o algoritmo proposto neste trabalho para o problema de caminhos mínimos é muito custoso. Dito isto, pode se tornar inviável para uso em algumas aplicações da vida real, porém este trabalho visa despertar a curiosidade referente a importância e utilidade deste tipo de análise.

Do tempo de execução, pelos testes realizados é possível constatar que o tempo de processamento do grafo de entrada e cálculo dos valores de centralidade para todos os caminhos mínimos, cresce rapidamente conforme o aumento do número de vértices. Importante destacar isto, pois corrobora a necessidade de se explorar algoritmos mais eficientes, inclusive, já existem propostas de algoritmos que calculam o valor da centralidade de caminhos mínimos por meio de amostragem \cite{alane2021}.

Da análise dos valores de centralidade dos caminhos mínimos. De forma intuitiva, já se percebe um comportamento e relação entre o tamanho do percurso de um caminho mínimo e seu valor de centralidade, em outras palavras, caminhos mínimos com um percurso maior tendem a ter valores de centralidade menores. Explicando, dado um caminho mínimo, ele só poderá estar contido como subcaminho em algum outro, se e somente se, este conter pelo menos a mesma quantidade de vértices, ou seja, quanto menor o percurso, mais possibilidades de este estar contido como subcaminho mínimo em outros e, consequentemente, maior o seu valor de centralidade. Em resumo, pode-se dizer que, quanto maior o percurso de um caminho mínimo, menor seu valores de centralidade e, de forma análoga, quanto menor o percurso de um caminho mínimo, mais provável de seu valor de centralidade ser maior.

Do comportamento do gráfico de distribuição dos valores de centralidade dos caminhos mínimos. O Gráfico~\ref{sec5:short_path_centrality_ex2} e o Gráfico~\ref{sec5:short_path_centrality_ex3} apresentam a disposição dos valores de centralidade ordenados de forma crescente, isto nos permite constatar que a curva é sempre crescente. Isto é importante destacar, pois indica que os caminhos mínimos tendem a ser muito centrais ("importantes") ou pouco centrais e, outro fator interessante disto, assim como supracitado, os caminhos mínimos dispostos mais à esquerda, os valores menores, referem-se aos caminhos mínimos com percurso maior, enquanto que, de forma análoga, os caminhos mínimos dispostos mais à direita dos gráficos, representam os percursos maiores.

Dito tudo isto, vale ressaltar um aspecto em adicional. Sabemos que existem diversas situações em que, dado um grafo, existe a possibilidade de ocorrer mais de uma opção de caminho mínimo entre um par de vértices, isto é um aspecto importante, pois, de forma lógica, os valores de centralidade possivelmente mudem. Em grafos pequenos este aspecto pode ser facilmente percebido e por isso é necessário assumir como verdade o retorno de um algoritmo de busca por caminhos mínimos, pois cada algoritmo pode tratar de forma diferente a busca. Existem algumas linhas de pesquisas que indicam que o problema da aleatorização de rotulagem tem impacto desprezível em grafos muito esparsos, entende-se disto que a aleatorização de rotulagem não é, necessariamente, um empecilho para o estudo da centralidade de caminhos mínimos em um grafo \cite{alane2021}.