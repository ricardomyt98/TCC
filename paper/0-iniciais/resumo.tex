\begin{resumo}
	O objetivo deste trabalho é, por meio de três tipos de gráficos de entrada, enfatizar a importância de se analisar a centralidade dos caminhos mínimos. Para isto, foi proposto o uso do algoritmo de Dijkstra para a busca de caminhos mínimos entre todos os pares de vértices em um grafo e para armazenamento e manipulação de dados, foi proposto o uso de dicionários como estrutura de dados mais comum para armazenar os caminhos mínimos com seus respectivos valores de centralidade. O que motiva este trabalho é o fato de que os grafos representam muito bem situações da vida real, como mapas, redes sociais, entre outras. A busca por rotas otimizadas em mapas é um exemplo do escopo de grafos que cobre o problema de caminhos mínimos. É muito interessante analisar os valores de centralidades dos caminhos mínimos em um grafo e, um exemplo disto, no tráfego de dados em uma rede cabeada, existem pontos (rotas) nesta infraestrutura que devem ter estruturas físicas mais robustas para suportar a alta transferência de dados e afins. Pesquisar a centralidade dos caminhos mínimos é uma excelente forma de analisar estes estudos de casos. Os resultados apresentados neste trabalho permitirão visualizar a teoria algorítmica proposta, juntamente com o comportamento da distribuição dos valores de centralidade. \newline \newline
\end{resumo}
