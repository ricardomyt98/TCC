\begin{resumo}
	O objetivo deste trabalho é o estudo da medida de centralidade dos caminhos mínimos em grafos. A centralidade de caminhos mínimos diz respeito à análise da importância (influência) que um caminho mínimo tem sobre o fluxo de dados em um grafo. Para o cômputo da medida de centralidade foi executado o algoritmo de Dijkstra para a busca de caminhos mínimos a partir de todos os vértices em um grafo e, para manipulação de dados, dicionários foram usados como estrutura de dados para armazenamento do percurso dos caminhos mínimos com seus respectivos valores de centralidade. Dicionários são estruturas de dados muito eficientes e convenientes para a busca e armazenamento de informações indexadas por chaves. O que motiva este trabalho é o fato que grafos comumente representam situações do mundo real, como mapas, redes sociais, entre outras. A busca por rotas otimizadas em mapas é um exemplo do escopo de grafos que cobre o problema de caminhos mínimos. Dito isto, é de interesse e relevância analisar os valores de centralidades dos caminhos mínimos em um grafo, um exemplo disto, no tráfego de dados em uma rede cabeada, existem pontos (rotas) nesta infraestrutura que devem ter estruturas físicas mais robustas para suportar a alta transferência de dados, as rotas centrais dizem respeito à centralidade de um caminho mínimo, visto que o tráfego de dados busca, normalmente, sempre a menor rota. Os resultados apresentados neste trabalho permitirão visualizar a teoria algorítmica proposta, juntamente com o comportamento da distribuição dos valores de centralidade. \newline \newline
\end{resumo}
