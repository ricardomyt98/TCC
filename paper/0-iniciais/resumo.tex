\begin{resumo}
	O objetivo deste trabalho é o estudo da medida de centralidade dos caminhos mínimos. Para o cômputo da medida de centralidade foi executado o algoritmo de Dijkstra para a busca de caminhos mínimos a partir de todos os vértices em um grafo e para armazenamento e manipulação de dados, foi utilizado o uso de dicionários como estrutura de dados mais comum para armazenar os caminhos mínimos com seus respectivos valores de centralidade. O que motiva este trabalho é o fato de que os grafos representam muito bem situações da vida real, como mapas, redes sociais, entre outras. A busca por rotas otimizadas em mapas é um exemplo do escopo de grafos que cobre o problema de caminhos mínimos. Com isso, torna-se muito relevante analisar os valores de centralidades dos caminhos mínimos em um grafo, um exemplo disto, no tráfego de dados em uma rede cabeada, existem pontos (rotas) nesta infraestrutura que devem ter estruturas físicas mais robustas para suportar a alta transferência de dados e afins. Os resultados apresentados neste trabalho permitirão visualizar a teoria algorítmica proposta, juntamente com o comportamento da distribuição dos valores de centralidade. \newline \newline
\end{resumo}
