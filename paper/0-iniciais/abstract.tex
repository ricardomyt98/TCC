\begin{abstract}
	The objective of this work is, through three types of input graphs, to emphasize the importance of the analyses of the centrality of the shortest paths. For this, it was proposed to use the Dijkstra algorithm to search for the shortest paths between all pairs of vertices in a graph and for data store and manipulation, it was proposed to use dictionaries as the most common data structure to store all the shortest paths and their respective values of centrality. What motivates this work is the fact that the graphs represent very well real-life situations, such as maps, social networks, among others. The search for optimized routes on maps is an example of the scope of graphs that covers the shortest path problem. It is very interesting to analyze the values of the shortest paths centralities in a graph and, as an example of this, in the traffic of data in a wired network, there are points (routes) in this infrastructure that these must have more robust physical structures to hold on the high data transfer and alike. Research the centrality of the shortest paths is an excellent way to analyze these case studies. The results presented in this work will allow us to visualize the proposed algorithmic theory, together with the behavior of the distribution of centrality values. \newline \newline
\end{abstract}
