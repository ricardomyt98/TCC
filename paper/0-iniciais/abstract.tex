\begin{abstract}
	The objective of this work is to study the measure of centrality of the shortest paths. For the computation of the centrality measure, the Dijkstra algorithm was used to search for shortest paths from all vertices in a graph and for data storage and manipulation, the use of dictionaries was used as the most common data structure to store the shortest paths with their respective values of centrality. What motivates this work is the fact that the graphs represent real life situations very well, such as maps, social networks, among others. Searching for optimized routes on maps is an example of the scope of graphs that covers the shortest path problem. With this, it becomes very relevant to analyze the values of centralities of the shortest paths in a graph, an example of this, in data traffic in a wired network, there are points (routes) in this infrastructure that must have more robust physical structures to support the high data transfer and the like. The results presented in this work will allow us to visualize the proposed algorithmic theory, together with the behavior of the distribution of centrality values. \newline \newline\end{abstract}
