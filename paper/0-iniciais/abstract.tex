\begin{abstract}
	The goal of this work is to study the centrality measure of shortest paths in graphs. The centrality of shortest paths concerns the analysis of the importance (influence) that a shortest path has on the data flow in a graph. For the computation of such centrality, the Dijkstra algorithm was used to search the shortest paths from all vertices in a graph. For data manipulation, dictionaries were used as a data structure for storing the path of the shortest paths with their respective values of centrality. Dictionaries are very efficient and convenient data structures for searching and storing key-indexed information. What motivates this work is the fact that graphs commonly well represent real-world situations, such as maps, social networks, among others. Searching for optimized routes on maps is an example of the scope of graphs that covers the shortest path problem. Therefore, it is of interest and relevance to analyze the centrality values of shortest paths in a graph, an example of this, in data traffic on a wired network, there are points (routes) in this infrastructure that must have more robust physical structures to support the With high data transfer, central routes refer to the centrality of a shortest path, since data traffic normally always seeks the shortest route. The results presented in this work will allow us to visualize the proposed algorithmic theory, together with the behavior of the distribution of centrality values. \newline \newline
\end{abstract}
