\graphicspath{{\currfiledir/images/}}

%----------------------------------------------------------------------------------------
%	Metodologia
%----------------------------------------------------------------------------------------
\chapter{Metodologia}
De um ponto de vista algorítmico e de estruturas de dados, os grafos são amplamente utilizados para representar redes em geral, sejam estas representações de mapas, redes sociais ou até mesmo o fluxo de influências numa rede bibliográfica. Com isto, existem muitas e das mais diversas linhas de pesquisas envolvendo algoritmos e teoria dos grafos. Os estudos vão desde análise de uma rota otimizada em um mapa à análise do surto de uma transmissão viral. Seguindo-se nesta linha, o estudo sobre a análise da centralidade de caminhos mínimos é bastante relevante em diversos cenários da vida real. Um exemplo, dado um mapa, existem rotas que sempre são mais comumente usadas por transportes de carga rodoviário, desta forma, espera-se que este trajeto demande de mais frequente manutenção. Este e muitos outros cenários nos levam a entender a importância do estudo e análise da centralidade de caminhos mínimos.

O problema de centralidades em grafos é um problema muito explorado e já existem diversas técnicas para análise da centralidade de vértices, porém a centralidade de arestas, caminhos mínimos, presente neste mesmo escopo pesquisa, ainda carece de estudos. É um problema extremamente interessante de exploração, pois representa um enorme leque de aplicações da vida real.

Neste trabalho foram implementados e executados, experimentos de conceitos introduzidos no trabalho de doutorado da Alane Marie de Lima e publicado em \cite{alane2021}. O problema da centralidade de caminhos mínimos é explorado no trabalho da Alane de um ponto de vista de aplicabilidade, pois o algoritmo e todo seu processo, já em teoria, são extremamente custosos em tempo de execução para grafos muito esparsos. A pesquisa dela busca métodos de análise da centralidade de caminhos mínimos de forma mais eficiente.

Dito isto, este trabalho irá mostrar a aplicação de um algoritmo exato em três exemplos de grafos. O objetivo central é investigar, mesmo que preliminarmente, a distribuição dos valores de centralidade de caminhos mínimos. Os grafos que serão utilizados neste trabalho não são direcionados.

Para apresentação visual e conceitual da metodologia serão utilizados: um grafo simples, com 5 vértices; o modelo de grafo do clube de karatê de Zachary, que é uma rede social de um clube universitário de karatê \cite{zachary2009}; a rede de jogos de futebol americano entre as faculdades da Divisão IA durante a temporada regular do outono de 2000 \cite{networkxfootball2021}.

%----------------------------------------------------------------------------------------
%	Algoritmo
%----------------------------------------------------------------------------------------
\section{Algoritmo}
Dados os grafos de entrada, a ideia geral do trabalho é obter todos os possíveis caminhos mínimos, isto é, executar o algoritmo de Dijkstra para cada um dos $n$ vértices do grafo. Como já mencionado, o algoritmo de Dijkstra, dado um vértice \emph{target}, constrói uma árvore de caminhos mínimos, sendo cada percurso da raiz até os nós da árvore, um caminho mínimo da raiz ao nó. Situações em que o grafo é conexo, a quantidade de caminhos mínimos de um vértice para todos os demais é de $n - 1$. Este processo será executado $n$ vezes, isto é, para todos os vértices do grafo.

O trabalho foi desenvolvido em \emph{Python 3}, com uso da biblioteca \emph{NetworkX} \cite{networkx2021}.

A função \emph{main()}, serve como chamada central de outras subfunções que fazem o tratamento dos dados e que serão explicadas em sequência. De forma geral, esta função faz a construção de um grafo (linhas 2 - 4), determina os caminhos mínimos para todos os vértices (linhas 5 - 8), faz a computação de centralidade dos caminhos mínimos de um grafo (linha 7) e exibe, na saída padrão, todos os caminhos mínimos com seus respectivos valores de centralidade (linha 8).

\begin{lstlisting}[caption={Função central para chamada de subfunções.}]
	def main() -> None:
		G = simple_graph_generator()
		G = karate_club_graph_generator()
		G = football_graph_generator()

		dijkstraTrees = get_dijkstra_trees_from_a_graph(G)
		shortestPaths = get_shortest_paths(dijkstraTrees)
		all_shortest_paths_centrality(shortestPaths)
		print_all_paths_and_centrality(shortestPaths)
\end{lstlisting}

Para a geração de grafos de entrada, meu código possui 3 funções: o Algoritmo~\ref{sec4:funcao_simple_graph_generator}~\emph{($simple\textunderscore graph\textunderscore generator()$)}, o Algoritmo~\ref{sec4:funcao_karate_club_graph_generator}~\emph{($karate\textunderscore club\textunderscore graph\textunderscore generator()$)} e o Algoritmo~\ref{sec4:funcao_rede_futebol}~\emph{($football\textunderscore graph\textunderscore generator()$)}. Estas funções usam a biblioteca \emph{NetworkX} para construção manual do grafo simples, chamada do grafo, já predefinido na biblioteca, do clube de karatê e chamada por fonte externa do grafo da rede de jogos de futebol americano.

\begin{lstlisting}[caption={Função para gerar um grafo simples.}\label{sec4:funcao_simple_graph_generator}]
	def simple_graph_generator():
		G = nx.Graph()
		G.add_edge(1, 2)
		G.add_edge(1, 3)
		G.add_edge(1, 5)
		G.add_edge(2, 3)
		G.add_edge(3, 4)
		G.add_edge(4, 5)
		return(G)
\end{lstlisting}

\begin{lstlisting}[caption={Função para gerar o modelo de grafo do clube de karatê de Zachary.}\label{sec4:funcao_karate_club_graph_generator}]
	def karate_club_graph_generator():
		G = nx.karate_club_graph()
		for v in G:
			print(f"{v:4} {G.degree(v):6}")
		return(G)
\end{lstlisting}

\begin{lstlisting}[caption={Função para gerar o modelo de grafo da rede de futebol americano.}\label{sec4:funcao_rede_futebol}]
	def football_graph_generator():
		url = "http://www-personal.umich.edu/~mejn/netdata/football.zip"
		sock = urllib.request.urlopen(url)  # open URL
		s = io.BytesIO(sock.read())  # read into BytesIO "file"
		sock.close()
		zf = zipfile.ZipFile(s)  # zipfile object
		txt = zf.read("football.txt").decode()  # read info file
		gml = zf.read("football.gml").decode()  # read gml data
		# throw away bogus first line with # from mejn files
		gml = gml.split("\n")[1:]
		G = nx.parse_gml(gml)  # parse gml data
		return G
\end{lstlisting}

Com os grafos de entrada criados, iniciaremos o processo de tratamento destes dados.

Dado um grafo de entrada, o Algoritmo~\ref{sec4:funcao_get_dijkstra_trees_from_a_graph}~(\emph{$get\textunderscore dijkstra\textunderscore trees\textunderscore from\textunderscore a\textunderscore graph()$}) trata os dados de entrada para passar por cada um dos vértices, gerando a árvore de Dijkstra. Estes dados são armazenados em um dicionário de caminhos mínimos, indexado pelos rótulos dos vértices.

\begin{lstlisting}[caption={Função para tratar os dados de um grafo de entrada e gerar um dicionário de árvores de Dijkstra, indexado pelos rótulos dos vértices do grafo.}\label{sec4:funcao_get_dijkstra_trees_from_a_graph}]
	def get_dijkstra_trees_from_a_graph(g: dict) -> dict:
		graphDict = {}
		for node in g:
			if node not in graphDict:
				graphDict[node] = single_source_dijkstra(g, node)[1]
		return graphDict
\end{lstlisting}

No próximo passo, o Algoritmo~\ref{sec4:get_shortest_paths_from_dijkstra_trees}~(\emph{$get\textunderscore shortest\textunderscore paths\textunderscore from\textunderscore dijkstra\textunderscore trees()$}) é uma etapa de refinamento dos dados previamente obtidos, isto é, ele pega os dados do dicionário de caminhos mínimos retornados pela função \emph{$get\textunderscore dijkstra\textunderscore trees\textunderscore from\textunderscore a\textunderscore graph()$} e faz operações de normalização textual, como retirada de colchetes nos nomes dos caminhos mínimos, por exemplo.

\begin{lstlisting}[caption={Função para refinamento do dicionário de caminhos mínimos.}\label{sec4:get_shortest_paths_from_dijkstra_trees}]
	def get_shortest_paths_from_dijkstra_trees(graphDict: dict) -> dict:
		pathDict = {}

		for i in graphDict.values():
			for j in i.values():
				path = str(j).replace("[", "").replace("]", "")
				if path not in pathDict and len(path) > 1:
					newPathInfo = PathInfo(0, len(j))
					pathDict[path] = newPathInfo

		return pathDict
\end{lstlisting}

Agora sim, tendo-se os dados refinados, esta etapa visa calcular a centralidades dos caminhos mínimos, Algoritmo~\ref{sec4:all_shortest_paths_centrality} (\emph{$all\textunderscore shortest\textunderscore paths\textunderscore centrality()$}). A centralidades dos caminhos mínimos do grafo é calculada da seguinte forma: para cada caminho mínimo $cm_1$, verifique em todos os demais caminhos mínimos se $cm_1$ está presente como subcaminho mínimo (\emph{linhas 2 - 5}); em caso positivo, contabilize $+1$ na centralidade de $cm_1$ (\emph{linha 6}); estas etapas são repetidas para todos os demais caminhos mínimos (\emph{linhas 2 - 6}); feito isto, agora a última etapa é referente a normalização dos valores (\emph{linhas 11 - 13}), pois é comumente visto valores de ponderação com valores entre $0$ e $1$, ou seja, quanto mais próximo de $1$, maior a centralidade do caminho mínimo.

\begin{lstlisting}[caption={Função para cálculo da centralidade dos caminhos mínimos.}\label{sec4:all_shortest_paths_centrality}]
def all_shortest_paths_centrality(pathDict: dict) -> dict:
	# Check if i is contained in j. If it is, plus one
	for i in pathDict:
		for j in pathDict:
			if i in j:
				pathDict[i].centrality += 1

	# Get the number of Dijkstra trees
	numTrees = len(pathDict)

	# Normalizing centrality values
	for path in pathDict:
		pathDict[path].centrality /= numTrees * pathDict[path].numNodes
\end{lstlisting}