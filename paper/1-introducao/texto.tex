\chapter{Introdução}

%=====================================================
\section{Justificativa}
As redes estão em todo lugar, desde a internet com a estrutura da World Wide Web, economia, geografia à transmissão de doenças, estes tipos de dados são comumente dispostos em grafos, pois são considerados um método muito conveniente para ilustrar visualmente as relações nos dados. O objetivo de um grafo é apresentar dados que são muito numerosos ou complicados para serem descritos de forma adequada no texto e em menos espaço.

Tendo-se que os grafos têm alta representatividade de dados da vida real, a manipulação e análise destes, tal como, a exploração de caminhos mínimos, tópico de discussão deste trabalho, resultam em inúmeras soluções e facilidades para aplicações na realidade. O problema de caminhos mínimos é encontrar um percurso entre dois vértices (ou nós) em um grafo de forma que a soma dos pesos de suas arestas constituintes seja minimizada, este conceito facilmente representa algumas situações reais como busca por rotas otimizadas em mapas, análise de mantenabilidade de rodovias centrais (rotas mais usadas), estudo de transmissibilidade de surtos de doenças (COVID-19), dentre outros. Dado estes exemplos, a busca pela centralidade dos caminhos mínimos nos permite analisar e definir quais são os meios mais comuns de propagação. No exemplo de rodovias, quais precisarão de maior monitoramento para manutenção, no caso de transmissibilidade viral, tentar mitigar os meios mais comuns de transmissão do vírus (rotas aéreas para regiões com surto da doença, por exemplo).


\section{Objetivos e motivação}
Este trabalho instiga o estudo e importância da centralidade de caminhos mínimos em um grafo. A análise de centralidade de vértices em grafos tem sido muito discutida e já possui alguns bons algoritmos teóricos e otimizados para isto. A computação da centralidade de caminhos mínimos é um direcionamento, deste mesmo escopo, que ainda carece de algoritmos de viável aplicação na vida real, pois é um processo extremamente custoso, seguindo-se da teoria direta para o cálculo do valor da centralidade. Dito isto, este trabalho irá passar o conceito da metodologia para computação da centralidade dos caminhos mínimos (algoritmo de força bruta), apresenta resultados obtidos, dados numéricos e gráficos visuais, para estudo de comportamento e ressalta pontos adicionais de exploração, como a aleatorização de rotulagem.


\section{Organização do trabalho}
De forma geral, o trabalho está dividido em 3 grandes partes: introdução dos conceitos básicos; apresentação da metodologia para a computação dos caminhos mínimos; apresentação dos resultados e suas respectivas análises e conclusões. A "introdução" visa definir os conceitos principais para o entendimento do algoritmo proposto neste trabalho, para a computação dos valores de centralidades dos caminhos mínimos, definições como, vértices, arestas, direcionamento, árvores, caminhos, dentre outros. O capítulo "metodologia" apresenta o processo passo a passo do algoritmo para o cálculo dos valores de centralidade dos caminhos mínimos, desde o formato de recepção de dados (dados de entrada)à saída (impressão na saída padrão) dos valores computados. E, por fim, a última parte, "resultados" e "conclusões", apresenta os dados obtidos pelo algoritmo, como as tabelas de centralidade, representação gráfica destes valores, análise de comportamento, juntamente com as conclusões.

%=====================================================
