%----------------------------------------------------------------------------------------
%	Introdução & Justificativa
%----------------------------------------------------------------------------------------
\chapter{Introdução}
\section{Justificativa}
As redes estão em todo lugar, desde a internet com a estrutura da World Wide Web, economia, geografia à transmissão de doenças, estes tipos de dados são comumente modelados por grafos, pois são considerados um método muito conveniente para ilustrar visualmente as relações nos dados.

Tendo-se que os grafos têm alta representatividade de dados da vida real, a manipulação e análise destes resultam em inúmeras soluções e facilidades para aplicações na realidade. O problema dos caminhos mínimos, tópico deste trabalho, é encontrar um percurso entre dois vértices (ou nós) em um grafo de forma que a soma dos pesos de suas arestas constituintes seja minimizada, este conceito facilmente representa algumas situações reais como busca por rotas otimizadas em mapas, análise de mantenabilidade de rodovias centrais (rotas mais usadas), estudo de transmissibilidade de surtos de doenças (COVID-19), dentre outros. Dado estes exemplos, a busca por caminhos mínimos mais importantes (caminhos mínimos de alta centralidade) nos permite analisar e definir quais são os meios mais comuns de propagação. No exemplo de rodovias, quais precisarão de maior monitoramento para manutenção, no caso de transmissibilidade viral, tentar mitigar os meios mais comuns de transmissão do vírus (rotas aéreas para regiões com surto da doença, por exemplo).

%----------------------------------------------------------------------------------------
%	Objetivos e motivação
%----------------------------------------------------------------------------------------
\section{Objetivos e motivação}
Este trabalho instiga o estudo da centralidade de caminhos mínimos em um grafo. A análise de centralidade de vértices em grafos tem sido muito discutida e já dispõe de diversos algoritmos cujo entendimento, tanto do ponto de vista teórico como prático, têm sido investigados na literatura \cite{williams2014} \cite{pettie2002}. No escopo de caminhos mínimos o algoritmo que computa os valores exatos de centralidade de caminhos mínimos é extremamente custoso, tornando-o inviável para aplicações em grafos muito grandes, justamente os que, maioritariamente, representam situações da vida real. Contudo, existem linhas de pesquisas para otimização deste processo de computação, o cálculo por amostragem \cite{alane2021} é um exemplo recentemente proposto para a solução do problema. Dito isto, este trabalho irá passar o conceito base para a computação dos valores de centralidade dos caminhos mínimos (algoritmo de força bruta), apresentando resultados, dados numéricos e gráficos visuais para estudo de comportamento e ressalta, adicionalmente, a questão da aleatorização de rotulagem.

%----------------------------------------------------------------------------------------
%	Organização do trabalho
%----------------------------------------------------------------------------------------
\section{Organização do trabalho}
De forma geral, o trabalho está dividido em três partes: introdução dos conceitos básicos; apresentação da metodologia para a computação da centralidade dos caminhos mínimos; apresentação dos resultados e suas respectivas análises e conclusões. A introdução visa definir os conceitos principais para o entendimento do algoritmo proposto neste trabalho, para a computação dos valores de centralidades dos caminhos mínimos, definições como, vértices, arestas, direcionamento, árvores, caminhos, dentre outros. O capítulo metodologia apresenta o processo passo a passo do algoritmo para o cálculo dos valores de centralidade dos caminhos mínimos, desde o formato de recepção de dados (dados de entrada)à saída (impressão na saída padrão) dos valores computados. E, por fim, a última parte, resultados e conclusões, apresenta os dados obtidos pelo algoritmo, como as tabelas de centralidade, representação gráfica destes valores, análise de comportamento, juntamente com as conclusões.