%----------------------------------------------------------------------------------------
%	Introdução & Justificativa
%----------------------------------------------------------------------------------------
\chapter{Introdução}
\section{Justificativa}
As redes são comuns em diversos tipos de aplicações, desde a internet com a estrutura da World Wide Web, economia, geografia à transmissão de doenças. Estes tipos de aplicações são comumente modeladas por grafos, pois estes são considerados um modelo matemático bastante conveniente para ilustrar visualmente as relações entre os dados.

Tendo-se em vista a alta representatividade de relações entre dados que os grafos proporcionam, a manipulação e análise destes, resultam em inúmeras soluções e facilidades para aplicações na realidade. O problema dos caminhos mínimos, tópico deste trabalho, é encontrar um percurso entre dois vértices (ou nós) em um grafo de forma que a soma dos pesos de suas arestas constituintes seja minimizada. Este conceito se aplica a situações de relevância como a busca por rotas mínimas em mapas, análise de manutenibilidade de rodovias centrais (rotas mais usadas), estudo de transmissibilidade de surtos de doenças (COVID-19), dentre outros. Dado estes exemplos, a busca por caminhos mínimos mais importantes (caminhos mínimos de alta centralidade) nos permite analisar e definir quais são os meios mais comuns de propagação de influência no fluxo de informações em um grafo. No caso de rodovias, destas, quais precisarão de maior monitoramento para manutenção; no caso de transmissibilidade viral, tentar mitigar os meios mais comuns de transmissão do vírus (rotas aéreas para regiões com surto da doença, por exemplo).

%----------------------------------------------------------------------------------------
%	Objetivos e motivação
%----------------------------------------------------------------------------------------
\section{Objetivos e motivação}
Este trabalho instiga o estudo da centralidade de caminhos mínimos em um grafo. O problema de caminhos mínimos para todos os pares de vértices dispõe de diversos algoritmos cujo entendimento, tanto do ponto de vista teórico como prático, têm sido investigados na literatura \cite{williams2014} \cite{pettie2002}. No escopo de caminhos mínimos, o algoritmo que computa os valores exatos de centralidade de caminhos mínimos é extremamente custoso, tornando-o inviável para aplicações em grafos muito grandes, justamente os que, maioritariamente, representam situações da vida real. Contudo, existem linhas de pesquisas que buscam alternativas estratégicas para computar esta medida de forma mais eficiente, o cálculo por amostragem \cite{alane2021} é um exemplo recentemente proposto para a solução do problema. Dito isto, este trabalho irá passar o conceito base para a computação dos valores de centralidade dos caminhos mínimos (algoritmo de força bruta), apresentando resultados, dados numéricos e gráficos visuais para estudo de comportamento.

%----------------------------------------------------------------------------------------
%	Organização do trabalho
%----------------------------------------------------------------------------------------
\section{Organização do trabalho}
De forma geral, o trabalho está dividido em três partes: introdução dos conceitos básicos; apresentação da metodologia para a computação da centralidade dos caminhos mínimos; apresentação dos resultados e suas respectivas análises e conclusões. A introdução visa definir os conceitos principais para o entendimento do algoritmo proposto neste trabalho para a computação dos valores de centralidades dos caminhos mínimos e para definições. A metodologia apresenta o processo passo a passo do algoritmo para o cálculo dos valores de centralidade dos caminhos mínimos, desde o formato de recepção de dados (dados de entrada)à saída (impressão na saída padrão) dos valores computados. O capítulo final apresenta os dados retornados pelo algoritmo, como as tabelas de centralidade, representação gráfica destes valores e análise de comportamento do algoritmo, juntamente com as conclusões.